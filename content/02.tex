% !TeX root = ..\main.tex

\section{Key-Value-Stores}

Key-value stores are considered one of the simplest databases and belong to the group of NoSQL databases. 
Compared to rational databases, which work with tables, key-value stores use the key-value method to store the data. \parencite{Key_Value_Datastore_Hazelcast}


\subsection{History of Key-Value-Stores}

The history of the key-value store is difficult to trace due to poor documentation. 
The first mention of the key-value method was in connection with Charles Babbage's analytical engine in 1837, where the key-value method was to be used to enter values into the analytical engine. 
The first implementation of the same concept in computer science, however, was implemented in 1953 in the form of hashing. 
Hashing implements key-value in the same way as databases do. 
In the field of databases, the first implementation was in 1979 in the form of Ordered Key-Value-Stores. 
Ordered key-value stores are key-value stores that sort keys. 
Now that the history of key-value stores has been explained in more detail, the main elements of the concept will be discussed.\parencite{babge_table_cards, hash_table_history, ordered_key_value_history}

\subsection{Working principle of Key-Value-Stores}

Key values are composed of two elements, a key and a value. 
The key element must fulfil the condition of being unique. 
This condition does not refer to the stored values, these do not have to be unique and can comprise different data types. 
The data types can range from integer to complex Qbjects. 
Data is thus always stored in with a value and a key. 
In addition to storing data, searches within the database are limited by the key-value method. 
All searches are always related to keys and cannot access values, so searches related to values are not possible as in rational databases. 
Now that the basics of key-value stores have been explained, the advantages and disadvantages compared to rational databases will be discussed next. \parencite{Key_Value_Datastore_Hazelcast, Key_Value_Datastore}

\subsection{Comparison of Key-Value-Stores to relational databases}

During the research, advantages and disadvantages of this database were identified. 
The first advantage of a key-value store is the high speed and performance compared to relational databases.
The second advantage is the possibility to scale the database efficiently. 
All advantages can be traced back to the simple structure of the database. 
In contrast to rational databases, key-value stores do not require uniform patterns. 
Thus, the data can be retrieved quickly and the databases can be extended to new servers. 
However, these characteristics are also a reason for the disadvantages of key-value stores. 
The first disadvantage relates to the problem of efficiently representing complex connections in the database. 
This disadvantage arises from the fact that data is only stored in key-value format and keys are only stored uniquely. 
Complex connections are much easier to represent in relational databases. 
The second disadvantage is the problem that no search queries can be made on the values. 
This disadvantage also stems from the key-value method. 
In order to clarify the advantages and disadvantages, the areas of application of this database will now be discussed in more detail. 
The field of application of key-value stores is limited compared to relational databases. 
This can largely be explained by the limitations of the structure. 
Generally, key-value stores are used when large amounts of data need to be accessed quickly. 
This is a typical case for applications such as shopping carts or session data. 
Besides these applications, key-value stores can be used for in-memory data caching. 
This type of use reduces reading and writing on weaker hard disks. 
The Hazelcast database system also uses this system to retrieve data faster. 
Hazelcast is discussed in more detail in the next section. \parencite{Key_Value_Datastore_Hazelcast, Key_Value_Datastore}