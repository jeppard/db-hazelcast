% !TeX root = ..\main.tex

\todo{Ersetzen mit Fernands Einleitung}
In the database engines Rating by \textcite{db-engines} Multi-model and NoSQL
databases are gaining popularity. We will focus on the Key-Value store
Hazelcast. Hazelcast is a distributed in-memory data grid, with a streaming
architecture. The goal of this seminar work is to get a better understanding of
the Hazelcast platform. To do so, we will implement a simple data structure
given by a previous lecture. Furthermore, a closer look at Hazelcast in relation
to the CAP-Theorem \parencite{Brewer2000} will be provided.

First, we will take a look at Key-Value-Stores in general, after that the
Hazelcast platform is described. In this section the Hazelcast is sorted into
the CAP-Theorem. After this we will take at the BASE (short for
\textbf{B}asically \textbf{A}vailable, \textbf{S}oft-state and \textbf{E}ventual
consistency) \parencite{Brewer2000} approach taken by Hazelcast
\parencite{HZfailure}. After that, we will take a look at Hazelcast in practice.
In this section we will discuss the installation of Hazelcast and, the
aforementioned example. At the end this seminar paper is a short
conclusion and outlook with recommendations when to use Hazelcast.