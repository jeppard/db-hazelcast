% !TeX root = ..\main.tex

In recent years, many areas have been changed by digitalization.
The digitalization of areas has led to many changes and improvements. The
possibilities of storing large amounts of data have become increasingly
important. In 2020, a total of 64.2 zettabytes of data were generated in the
world. Forecasts for 2025 expect this to increase to 181 zettabytes of data. In
order to cope with these large amounts of data, new database management systems
are constantly being developed and improved. However, rational databases such as
MySQL are not suitable for storing this amount of data efficiently. For this
reason, NoSQL database solutions are becoming increasingly important to deal
with these data volumes. In this report, the Key-Value-Stores are analysed in
more detail and explained using the example of Hazelcast. In the following
section, the general concept of key-value databases is discussed.
\parencite{data_development_2012_2025, Key_Value_Datastore} After that the
Hazelcast platform is described. In this section the Hazelcast platform is sorted into
the CAP-Theorem. After this we will take at the BASE (short for
\textbf{B}asically \textbf{A}vailable, \textbf{S}oft-state and \textbf{E}ventual
consistency) \parencite{Brewer2000} approach taken by Hazelcast
\parencite{HZfailure}. After that, we will take a look at Hazelcast in practice.
In this section we will discuss the installation of Hazelcast and, the
aforementioned example. At the end this seminar paper is a short conclusion and
outlook with recommendations when to use Hazelcast.