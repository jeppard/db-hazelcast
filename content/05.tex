% !TeX root = ..\main.tex

\section{Evaluation of Hazelcast}
\todo{}
\subsection{Advantages of Hazelcast}
\todo{}
\subsection{Disadvantages of Hazelcast}
\todo{}
\subsection{Lessons learned}
This paragraph contains a short summary of the lessons learned during the
evaluation of Hazelcast in Practice. First, Hazelcast is easiest to
install via docker. Secondly the Hazelcast CLI is easy to use with common SQL
like syntax, but has some limitations and differences. Mainly in Hazelcast the
common concept is a map (Mapping) instead of a table. As a user you have to
create mappings and not tables. After that you should take a look at the
Hazelcast Client APIs to use Hazelcast from within your application. These
Clients are currently available for Java, .NET, C++, Node.js, Python and Go
\parencite{noauthor_clients_nodate}.